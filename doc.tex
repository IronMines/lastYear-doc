\documentclass[a4paper, 11pt]{article}
\usepackage[utf8]{inputenc} 
\usepackage[T1]{fontenc}
\usepackage{lmodern}
\usepackage{graphicx}

\usepackage[french]{babel}

\usepackage{xcolor}
\usepackage{listings}
\input{macros}
 
\begin{document}

\section{Introduction}

Le projet Ironmines a été créé pour répondre offrir une interface de
programmation pour faire évoluer le robot Kobuki. L'objectif est
bien-sur de le faire concourir à la coupe de France de Robotique.

\section{Installation}

À partir d'une distribution Raspbian vierge il faudra installer les
dépendances suivantes :
\begin{itemize}
\item ROS - Kobuki
\item Rosserial (pour communiquer avec l'Arduino)
\item PhaROS - Ironmines
\end{itemize}

\subsection{ROS - Kobuki}

Tout d'abord il faut installer ROS sur le Raspberry : 

\begin{lstlisting}
  sudo sh -c 'echo "deb http://64.91.227.57/repos/rospbian wheezy main" > /etc/apt/sources.list.d/rospbian.list'
  wget http://64.91.227.57/repos/rospbian.key -O - | sudo apt-key add -

  sudo apt-get update

  sudo apt-get install ros-groovy-ros-comm
  sudo rosdep init
  rosdep update

  echo "source /opt/ros/groovy/setup.bash" >> ~/.bashrc
  echo "export ROS_HOSTNAME=localhost" >> ~/.bashrc
  source ~/.bashrc
\end{lstlisting}

Puis installer les noeuds pour communiquer avec le Kobuki : 

\begin{lstlisting}
  sudo apt-get install ros-groovy-kobuki-node 
  sudo apt-get install ros-groovy-kobuki-msgs
  sudo apt-get install ros-groovy-kobuki-ftdi
\end{lstlisting}

On pourra alors tester l'installation en exécutant dans un terminal :

\begin{verbatim}
  roscore
\end{verbatim}

Et dans un autre terminal : 

\begin{verbatim}
  roslaunch kobuki_node minimal.launch
\end{verbatim}

Si tout ce lance correctement, on peut passer à la suite.

\subsection{Rosserial}
  
\section{Architecture}

D'un façon générale, Pharo fait office d'intelligence. ROS ne fait que
transmettre les message provenant de l'Arduino et du Kobuki au travers
des noeuds Rosserial et Kobuki\_node.
\begin{center}
  \includegraphics[width=\linewidth]{./architecture.jpg}
  \caption{Architecture générale}
  \label{archi_generale}
\end{center}
\subsection{Arduino}
Le code pour l'arduino se trouve a l'Url
:\\ https://github.com/mattonem/ironmines-arduino.git.

Basiquement, on utilise la librairie fournie par Rosserial pour
pouvoir publier différents topics sur le port serie ainsi que souscrire
a d'autre. On publie les données relatives au contacteur de démarrage
et au capteurs ultrasons. On souscrit aux différents topics concernant
les actions a effectuer (activer le canon, bouger le bras
retractable).

\subsubsection{Subscribed Topics}

\begin{itemize}
\item  todo
\end{itemize}

\subsubsection{Published Topics}

\begin{itemize}
\item \texttt{/sonar/<i> (std\_msgs/UInt16)}\\
  Donne la distance au mur en face du sonar i.
\item \texttt{/startTrigger (std\_msgs/Bool)}\\ 
  Donne les fronts montant et descendant pour le cordon de demarrage.
\end{itemize}

\end{document}
