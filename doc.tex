\documentclass[a4paper, 11pt]{article}
\usepackage[utf8]{inputenc} 
\usepackage[T1]{fontenc}
\usepackage{lmodern}
\usepackage{graphicx}

\usepackage[french]{babel}

\usepackage{xcolor}
\usepackage{listings}
\input{macros}
 
\begin{document}

\section{Introduction}

Le projet Ironmines a été créé pour répondre offrir une interface de
programmation pour faire évoluer le robot Kobuki. L'objectif est
bien-sur de le faire concourir à la coupe de France de Robotique.

\section{Installation}

À partir d'une distribution Raspbian vierge il faudra installer les
dépendances suivantes :
\begin{itemize}
\item ROS - Kobuki
\item Rosserial (pour communiquer avec l'Arduino)
\item PhaROS - Ironmines
\end{itemize}

\subsection{ROS - Kobuki}

Tout d'abord il faut installer ROS sur le Raspberry : 

\begin{lstlisting}
  sudo sh -c 'echo "deb http://64.91.227.57/repos/rospbian wheezy main" > /etc/apt/sources.list.d/rospbian.list'
  wget http://64.91.227.57/repos/rospbian.key -O - | sudo apt-key add -

  sudo apt-get update

  sudo apt-get install ros-groovy-ros-comm
  sudo rosdep init
  rosdep update

  echo "source /opt/ros/groovy/setup.bash" >> ~/.bashrc
  echo "export ROS_HOSTNAME=localhost" >> ~/.bashrc
  source ~/.bashrc
\end{lstlisting}

Puis installer les noeuds pour communiquer avec le Kobuki : 

\begin{lstlisting}
  sudo apt-get install ros-groovy-kobuki-node 
  sudo apt-get install ros-groovy-kobuki-msgs
  sudo apt-get install ros-groovy-kobuki-ftdi
\end{lstlisting}

On pourra alors tester l'installation en excutant dans un terminal :

\begin{verbatim}
  roscore
\end{verbatim}

Et dans un autre terminal : 

\begin{verbatim}
  roslaunch kobuki_node minimal.launch
\end{verbatim}

Si tout ce lance correctement, on peut passer à la suite.

\subsection{Rosserial}
  
\section{Architecture}




\end{document}
